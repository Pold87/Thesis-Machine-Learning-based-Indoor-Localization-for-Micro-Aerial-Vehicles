\section{Pillar I: Training Set Generation}
\label{sec:mapping}

A main idea of the proposed method is to shift computational effort to
an offline phase, before the MAV is used on a repetitive task. This
shift is accomplished by using a light-weight machine learning
approach instead of a computationally more complex algorithm, such as
homography finding based on salient image keypoints.

Supervised machine learning needs a training set to find a mapping
from features to target values. In a computer vision-task, the
features represent image characteristics, such as average pixel values
or the amount of oriented gradients.

The proposed method is based on
histograms of textons. To create the training set, a convenient way is
to align image features---that is, the texton histograms---with
high-precision position estimates from a motion tracking system. 
These
systems can track rigid bodies at a high frequency within an error of
few millimeters. This allows to create high-quality training data
sets, where even low quality images can be mapped to the corresponding
$x,y$-position. 

This is achieved by saving the texton histogram on the MAV’s hard disk. The corresponding position from the motion tracking system is broadcast to the UAV via the ground station. The ground station receives the positions from the motion tracking system. 
The major disadvantages of the approach are that
motion tracking systems are usually expensive and time-consuming to
move to different environments.

As an alternative, we sought a low-budget and more flexible
solution. While the homography-based approach (Section xxx) shows high matching quality
for non-blurry images, it suffers from lacking robustness to noise
and the high processing time requirements. To exploit the advantages,
and straighten out the disadvantages, we used the approach in a preprocessing
step to obtain labeled training data of a known environment. This
allows to shift computational effort from the flight phase to a
pre-flight phase---paving the way for autonomous flights of MAVs with
limited processing power. It is based on two associated cornerstones
that are performed in an offline phase: the creation of an
\emph{orthomap} and finding a homography. The required image dataset
for both cornerstones can be obtained by using in-flight pictures of a
manual flight or by taking pictures with a hand-held camera.