For creating a map, the images from the dataset have to be stichted
together to get a hyperspatial image of the scene. The stitched image has a higher resolution than the
single images and contains a greater range of detail. The stitching
step includes challenges: a subset of the recorded images might be
distorted and perspective
transformations can impede the stitching process. Certain software packages allow for orthorectifying the images
 by estimating the most probable viewing angle based on
the set of all images. However, since a downward-looking camera is attached to
the UAV, most images will be roughly aligned with the z-axis, given
slow flight.