\subsection{Texton Dictionary Generation}
\label{sec:text-dict-gener}

For learning a suitable dictionary for an environment, image patches were clustered. The resulting cluster centers---the prototypes of the
clustering result---are the textons~\cite{varma2003texture}.  The clustering was performed using a competitive learning scheme with a winner-take-all strategy. 
In the beginning, the dictionary is initialized with 20 random patches from the first image, which form the first guess for cluster centers. Then, new images patches are extracted and  compared to each texton in the tentative dictionary using the Euclidean distance. The most similar texton to the current patch is declared as the ``winner". This texton is then adapted to be more similar to the current patch, by calculating the difference in pixel values between the texton and the current images patch, and updating the texton with a learning rate of $\alpha = 0.02$. The first 100 images of each dataset were used to generate the dictionary. For each image, 1000 randomly selected image patches of size $w \times h = 6 \times 6$ were extracted, yielding $100,000$ image patches in total that were clustered.  An example of a learned dictionary can be found in Figure~\ref{fig:dictionary}.
Different maps and environmental settings require different textons. If one would use the same dictionary for each map, it might happen that the histogram has only a few non-zero elements, and thus, cannot represent the variance in the map. While we set the number of textons to 20 for all maps, this parameter is also map-dependent, and ideally, if time allows, could be adapted to the given map. 