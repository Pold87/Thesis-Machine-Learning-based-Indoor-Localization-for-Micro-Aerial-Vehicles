This allows to make use of the information in all $k$ neighbors and
keep track of a multimodal distribution. While keeping track of a
multimodal distribution allows for incorporating several possible
states of the system, the system needs one best position estimate for guidance and stabilization after each iteration. Using a weighted average of the modes would again introduce the
problem, that the weighted average could fall into a low density
region. Therefore, the maximum a posteriori estimate, as described in
\cite{driessen2008map} was used. This approach uses the following
formula to obtain the MAP estimate:

Therefore, the final position estimate is equal to the position of one
of the particles.

\begin{align}
  s_k^{MAP}  &= \text{arg}\,\max_{s_k}{p(s_k \mid Z_k)}\\
             &= \argmax_{s_k}{p(z_k \mid s_k) p(s_k \mid Z_{k-1})} 
\end{align}

Therefore, the MAP estimator in our case is

\begin{align}
s_k^{MAP} = \argmax_{s_k} \lambda^M(s_k)
\end{align}
with

\begin{align}
\lambda^M(s_k) = p(z_k \mid s_k) \sum_{j=1}^Mp(s_k \mid s_{k-1}^j)w^j_{k-1}
\end{align}

This function is now only evaluated at a finite, chosen number of
states, the particles, using

\begin{align}
\hat{s}_k^{MAP} = \argmax_{s_k \in \{s_k^i \mid i=1,\ldots,N\}} \lambda^M(s_k)
\end{align}

In this formula, $p(z_k \mid s_k)$ is the likelihood of the particle
$s_k$ given the current measurement $z_k$. In our setting, this
probability is equal to the weight of the particle $z_k$, therefore
$p(z_k \mid s_k) = w^i_k$.

The estimation of \emph{uncertainty} is a core part of the proposed
approach, being important for safety and accuracy. Therefore,
uncertainty was modeled using the spread of the particles.

In every time step, the particles of the filter get updated based on
the optical flow estimates. These estimates are noisy, as illustrated
in Figure~\ref{fig:edgeflow}. Additionally, optical flow estimates
aggregate noise over time, since each estimate is dependent on the
previous one, leading to drift (\emph{relative position
  estimates}). In contrast, the machine learning-based method makes
independent predictions. While this allow for avoiding accumulating
errors, the predictions do not dependent on each other, and might be
`jumping' between two points. To combine the advantages of both
methods, and leverage out the disadvantages, the particle filter is
used.

An idea was to include the similarity to the neighbors as confidence
value, thus reducing the measurement noise, if a high similarity
between current histogram and a training histogram is
achieved. However, we found no correlation between these
variables. Figures~\ref{fig:cor_sim_confi} and \ref{fig:cor_sim_confi_y} displays the
dependence structure.