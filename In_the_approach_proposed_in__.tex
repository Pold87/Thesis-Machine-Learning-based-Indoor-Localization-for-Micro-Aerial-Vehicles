In the approach proposed in this thesis, computational power will be
shifted to an offline training phase to achieve high-speed during live
operation. In contrast to visual SLAM frameworks, this project considers
scenarios in which the environment is known beforehand or can be even
actively modified. The environment is non-dynamic and planar, therefore, the UAV will make use of texture on the bottom or ceiling of the environment. This opens the door for improving the proposed
algorithm by changing the map. On the basis of desired characteristics
of a given map, an evaluation technique was developed that determines
the suitability of an environment for the proposed approach. This
technique allows for spotting distant regions with similar image
features, which could lead to deteriorated performance. The evaluation
can be performed using a given map image or recorded images during
flight. In the former case, synthetic images will be generated from
the map image that simulate images taken during flight.