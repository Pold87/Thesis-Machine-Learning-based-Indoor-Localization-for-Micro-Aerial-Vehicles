\section{Texton-based Approaches}
\label{sec:textonbasedapproaches}

Textons are small characteristic image patches that can be used as image features.
\citeauthor{varma2005statistical} originally introduced textons for
classifying different textures, showing that they outperform
computationally more complex algorithms, like Gabor
filters~\cite{varma2005statistical}.

Instead of convolving a window over the entire image, the filters can be applied at randomly sampled image positions, leading to similar texton histograms compared to the histograms when full sampling is used. The choice of the number of samples allows for modifying the computational effort, resulting in a
trade-off between accuracy and execution frequency. A disadvantage is
that it discards all information about the spatial arrangement of
textons---it does not make use of the \emph{Where} of the information,
just of the \emph{What}, which might result in different areas with
similar histograms.

\citeauthor{de2009design} use the texton approach for distinguishing
between three height classes during flight~\cite{de2009design}. Using
a nearest neighbor classifier, their approach achieves a height
classification error of approximately 22\,\% on a hold-out test set.
This enables a flapping-wing MAV during an experiment to roughly hold
its height. In another work, \citeauthor{de2012appearance} introduce
the \emph{appearance variation cue} for estimating the proximity to
objects~\cite{de2012appearance}. Since closer objects should have less
variation, these objects should appear less varied. Using this method,
their MAV is successfully able to avoid obstacles in a $5m \times 5m$
office space and achieves an AUC of up to .97 on rather distorted
images. Additionally, it performs better than or, at least, equal to
optical flow estimations.