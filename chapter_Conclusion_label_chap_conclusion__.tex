\chapter{Conclusion}
\label{chap:conclusion}

This thesis presented a novel approach for lightweight indoor
localization of MAVs. We pursued an onboard design without the need of
an additional ground station to foster flexibility and autonomy. The
conducted on-ground and in-flight experiments underline the real-world
applicability of the system. Promising results were obtained for
position estimates and accurate landing in the indoor environment.

The approach is based on three pillars that we identified for indoor
localization for MAVs. The first pillar shifts computational effort
from the flight phase to an offline preprocessing step. This provides
the advantages of sophisticated algorithms, without affecting
performance during flight. The second pillar states that during flight
algorithms should be used that can trade off speed with accuracy. This
allows their use on a wide range of models, from pico-drones to MAVs
with a wing-span of over one meter. Examples of these adaptable
algorithms are the texton-based approach and the particle filter. The
third pillar is a known---and possibly---modifiable environment. This
knowledge and flexibility allows for predicting and improving the
quality of the approach.
%In contrast to \textsc{slam} frameworks, in
%which the task is to simultaneous mapping and localization, the
%presented approach is intended for various repetitive indoor
%activities.
