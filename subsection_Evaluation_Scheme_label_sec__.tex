\subsection{Evaluation Scheme}
\label{sec:evaluationscheme}

The performance of a method might largely depend
on the environment it is used in. The evaluation of a map is
difficult, since the obtained histograms during a real flight depend
on many factors: motion blur, distance to the map and rotations
proportional to the map.

Therefore, we propose an initial evaluation scheme for given
maps. This scheme assigns a global fitness value to a given map,
proportional to the expected accuracy if it is used in the physical
world. Additionally, it allows to inspect the given map and detect the
regions that are responsible for the overall fitness value.

In the first step of the map evaluation procedure, $N$ different
patches of a given map are generated using the tool \emph{draug}
(Section~\ref{sec:draug}). We propose the following loss function
($L$) for evaluating a given map ($\mathcal{M}$):

\begin{align}
  L(\mathcal{M}) &= \sum_{i = 1}^{N} \sum_{j = 1}^{N} \ell(d_a(h_i, h_j), d_e(h_i, h_j))
\end{align}

\begin{align}
  \ell(x, y) &= x - y\\
  d_a(h_i, h_j) &= \text{cosine\_similarity}(h_i, h_j)\\
  d_e(h_i, h_j) &= f_X(pos_i) = f_X(x_i, y_i)\\
\end{align}

\begin{align}
\mu = pos_j = (x_j, y_j)\\
\Sigma =
  \begin{bmatrix}
    \rho & 0\\
    0 & \rho\\
  \end{bmatrix}
\end{align}

The idea behind the global loss function $L$ is that histograms in closeby areas
should be similar and the similarity should decrease the further away
two positions are. This is modeled as a 2-dimensional Gaussian with 0
covariance (Figure~\ref{fig:model}). The variance is depended on the
desired accuracy ($\rho$): the lower the variance, the more punctuated
a certain location is but also the higher the risk that a totally
wrong measurement occurs. The following visualization are based on
color histograms (and not texton histograms) for easier visual
analysis.