Widespread applications, from surveillance to search and rescue operation, make Micro Air Vehicles (MAVs) flexible platforms.
However, due to their small size, MAVs have limited processing power and cannot
fall back to standard localization techniques. To address this
issue, this thesis describes an efficient vision-based onboard localization
technique. Using a machine
learning approach, $x,y$-coordinates are estimated within a known, modifiable indoor
environment. The computational power of the approach is scalable to to different platforms, trading off speed and accuracy.

The development, software and hardware
implementation, and results of the localization system are presented. The proposed system uses 
textons as image features. Textons are small, characteristic image patches and their histograms are used as features for the $k$-Nearest Neighbors ($k$NN)
algorithm. The outputs of this regression technique--multiple possible
$x, y$-coordinates are used in a particle filter to neatly aggregate
and smooth the estimates. The estimates of the computer vision-based
system are compared to the ground truth in five on-ground and two
in-flight experiments. Promising results were obtained for all tasks:
waypoint navigation, accurate landing, and stable hovering in the
indoor environment.

To predict the performance of the proposed algorithm in different environments, an evaluation
technique is developed that compares actual histogram similarities to
ideal histogram similarities based on distance between the underlying
positions. The technique assigns a loss value to a given set of
images, allowing to compare different environments. Therefore, it
allows for spotting difficult or ambiguous locations or safe landing
spots. Additionally, a tool is presented that creates synthetic images
that could be taken during an actual flight. The synthetic images are
used to compare 46 possible maps---images with a high resolution. The
best one, the worst one and the one with median loss were printed out
to compare their performance in the real-world. In fact, the results
could be replicated on these maps.

The approach is based on three pillars: (i) a shift of processing power to an pre-flight phase to pre-compute
computationally complex steps, (ii) lightweight and adaptable algorithms to ensure real-time performance and portability to different platforms,
(iii) modifiable environments that can be tailored to the proposed algorithm. 