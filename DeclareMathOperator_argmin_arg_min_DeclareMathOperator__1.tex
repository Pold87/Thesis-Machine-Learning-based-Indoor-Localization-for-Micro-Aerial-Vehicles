\DeclareMathOperator*{\argmin}{arg\,min}
\DeclareMathOperator*{\argmax}{arg\,max}

\section{Hardware and Software of the Platform}
\label{sec:hardware}

In a first setting, the commercially available Parrot AR.Drone2.0 was
equipped with an Odroid XU-4 single board computer, a Logitech 525 HD
webcam, a WiFi module, and a USB connection between the Odroid board
and the AR.Drone2.0 flight controller to send and receive data between
the two processors. Figure~\ref{fig:comparison} shows the setup.

The Odroid processor has a full operating system (Ubuntu 15.04) and
allows to run arbitrary Linux software. While this setup allowed to
test algorithms on the rather powerful Odroid processor, the
additional weight resulted in unstable flight performance. Therefore,
we decided to modify the system to run it directly on-board of the UAV
and not on an external board such as the Odroid processor. This
removed the need for additional payload and made flight performance
very stable. Also, it reduced the need to buy and attach an external
processor and a point of failure. Another advantages is that the
framework can be easily ported to any UAV supported by the paparazzi
software. The disadvantages are that the on-board processors of many
MAVs have a lower performance than the Odroid processor. Additionally,
it required us to port the software from the high-level language
Python to the low-level language C without being able to rely on many
existing software libraries.

We decided to use the quadcopter \emph{Parrot Bebop Drone}
as prototype for all our tests. This UAV allow for navigating in
arbitrary directions without changing its yaw angle. Additionally,
several models show stable flight behavior, resulting in noise-free
images. It is equipped with a that allows for approximately 11 minutes
flying time. The UAV's dimensions are $28 \times 32 \times 3.6$\,cm
and it weighs 400\,g.

The UAV is equipped with two cameras: a front camera and a
downward-looking bottom camera. The proposed approach makes use of the
bottom camera only. This camera has a resolution of $640 \times 480$
pixels with a frequency of 30 frames per second. The UAV's processor
is a Parrot P7 dual-core CPU Cortex A9 with a tact rate of
800\,Mhz. It is equipped with 8~GB of flash memory. It runs a Linux
operating system. The full specifications of the UAV can be found at
\url{http://www.parrot.com/products/bebop-drone/}.

The original Bebop SDK was replaced with Paparazzi. Paparazzi is an
``open-source drone hardware and software project encompassing
autopilot systems and ground station software''. Its modular approach
allows for combining functions regarding stabilization, localization,
and control of UAVs.
The proposed approach is implemented as module in Paparazzi's computer
vision framework. Since low-level routines like accessing camera
information or attitude control for different platforms are already
implemented in Paparazzi, the proposed module can be readily used
across different platforms. Modules are written in the C programming
language and are cross-compiled on the host PC to make them suitable
for the UAV's processor. Afterwards, they are uploaded to the
microprocessor of the UAV, allowing for autonomous execution of the
compiled software. A downlink connection---from the UAV to the
groundstation---allows for monitoring the state of the aircraft (e.g.,
speed, altitude, position, battery status).

While the goal of the proposed system is to achieve an autonomous
system, monitoring the system is important for evaluation and safety
considerations. Figure~\ref{fig:gcs} shows the default ground control
station of Paparazzi.
For collecting data, real-time visualization of the position estimates
and enabling future users to keep track of the system states, a
graphical user interface (GUI) has been developed in the scope of this
thesis. The GUI is displayed in Figure~\ref{fig:gui}.

\begin{figure}[h!]
\begin{center}
\includegraphics[width=0.7\columnwidth]{figures/gui/default_figure}
\caption{{\label{fig:gui} Screenshot of the GUI. It displays
    the position esimate of the proposes framework, the ground truth
    position based on OptiTrack, the texton histogram, the Euclidean
    distances between OptiTrack and the system, and the uncertainty of
    the system (based on the spread of particles).%
}}
\end{center}
\end{figure}

The GUI displays the position estimate of the proposed framework, the
ground truth position based on OptiTrack, the texton histogram, the
Euclidean distances between OptiTrack and the system, the uncertainty
of the system, and the correlation between the uncertainty and the
Euclidean distance between the Optitrack and treXton estimates. The
GUI can be accessed using the \emph{Tool} menu in paparazzi.


\section{Pillar I: Training Set Generation}
\label{sec:mapping}

A main idea of the proposed method is to shift computational effort to
an offline phase, before the MAV is used on a repetitive task. This
shift is accomplished by using a light-weight machine learning
approach instead of a computationally more complex algorithm, such as
homography finding based on salient image keypoints.

Supervised machine learning needs a training set to find a mapping
from features to target values. In a computer vision-task, the
features represent image characteristics, such as average pixel values
or the amount of oriented gradients.

The proposed method is based on
histograms of textons. To create the training set, a convenient way is
to align image features---that is, the texton histograms---with
high-precision position estimates from a motion tracking system. 
These
systems can track rigid bodies at a high frequency within an error of
few millimeters. This allows to create high-quality training data
sets, where even low quality images can be mapped to the corresponding
$x,y$-position. 

This is achieved by saving the texton histogram on the MAV’s hard disk. The corresponding position from the motion tracking system is broadcast to the UAV via the ground station. The ground station receives the positions from the motion tracking system. 
The major disadvantages of the approach are that
motion tracking systems are usually expensive and time-consuming to
move to different environments.

As an alternative, we sought a low-budget and more flexible
solution. While the homography-based approach (Section xxx) shows high matching quality
for non-blurry images, it suffers from lacking robustness to noise
and the high processing time requirements. To exploit the advantages,
and straighten out the disadvantages, we used the approach in a preprocessing
step to obtain labeled training data of a known environment. This
allows to shift computational effort from the flight phase to a
pre-flight phase---paving the way for autonomous flights of MAVs with
limited processing power. It is based on two associated cornerstones
that are performed in an offline phase: the creation of an
\emph{orthomap} and finding a homography. The required image dataset
for both cornerstones can be obtained by using in-flight pictures of a
manual flight or by taking pictures with a hand-held camera.

For creating the orthomap, the images from the dataset have to be stichted
together to get a hyperspatial image of the scene. By orthorectifying
these pictures, different camera angles can be straightened out,
yielding seamless stitches. This results in a map of the
environment. The stitched image has a higher resolution than the
single images and contains a greater range of detail. The stitching
step includes challenges: a subset of the recorded images might be
distorted and each image must be orthorectified, otherwise perspective
transformations could hinder the stitching process. The images can be
orthorectified by estimating the most probable viewing angle based on
the set of all images. Since a downward-looking camera is attached to
the UAV, most images will be roughly aligned with the z-axis, given
slow flight.

For both steps, the software Agisoft
PhotoScan~\cite{agisoft2013agisoft} was used. The stitching process
can be time-consuming and error-prone. In some environments, an image
from a top view point can be taken capturing the entire area with
high-resolution camera, circumventing the need for stitching together
multiple images. Yet another method starts with an existing image and
modifies the environment accordingly---for example by painting the
floor or printing posters--to correspond to the image. In all cases,
we will describe the environment as \emph{map image}. 

Keypoints of the current image and the map image are detected and
described using the SIFT algorithm. This is followed by a matching
process, that identifies corresponding keypoints between both
images. These matches allow for finding a homography between both
images. For determining the $x, y$-position of the current image, the
center of it is projected on the reference image using the homography
matrix. The pixel position of the center in the reference image can be
used to determine the real world position by transforming the pixel
coordinates to real-world coordinates, based on the scale factors
$C_x$ and $C_y$, with $C_x = \frac{width(R)}{width(I)}$ and
$C_y = \frac{height(R)}{height(I)}$, where $W$ is the real-world
representation and $I$ the digital pixel image. This yields a dataset
of images, labeled with $x, y$ coordinates and the number of
matches. This process already introduces noise into the dataset, since
SIFT can have wrong and inaccurate matches.

\begin{figure}[h!]
\begin{center}
\includegraphics[width=0.7\columnwidth]{figures/map/default_figure}
\caption{{\label{fig:orthomap} This figure shows
    the created orthomap of a rather texture-rich floor. It is
    stitched together out of over 100 single images and represents a
    real world area of approximately $10\times10$ meters. Image
    distortions, non-mapped areas, and slightly skewed seams at
    several points are visible.%
}}
\end{center}
\end{figure}

\begin{figure}[h!]
\begin{center}
\includegraphics[width=0.7\columnwidth]{figures/sift/default_figure}
\caption{{\label{fig:homography} The
    image shows connected keypoint matches between the reference image
    on the left and the query image on the right. The green lines
    connect corresponding keypoints.%
}}
\end{center}
\end{figure}

In all cases, the result will be a labeled dataset of images and corresponding $x,y$-positions. The $x,y$-positions are of different quality depending on the used technique: orthomap-based, poster-based, or motion tracking-based. 


\section{Pillar II: Texton-based Approach}
\label{sec:textons}

In this section, the core of the proposed algorithm, the
implementation of the proposed texton framework is described.  The
histograms of textons are used as features for the $k$-Nearest
Neighbors ($k$NN) algorithm. The outputs of this regression technique
are possible $x,y$-coordinates for a given image.

In the following pseudo code, $M$ represents the number of particles,
$z_t^x$ the output of the texton framework at time $t$, and $f_t^x$
the estimated flow at time $t$.


Importantly, for generating the training dataset, no subsampling was
used. Therefore, the training dataset will not show any variance, if
the same images were used.

