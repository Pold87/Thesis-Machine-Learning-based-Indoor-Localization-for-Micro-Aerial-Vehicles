\subsection{Homography Determination \& Keypoint Matching}
\label{sec:keypointmatching}

A standard approach for estimating camera pose is detecting and
describing keypoints of the current view and a reference image, using
algorithms such as \textsc{Sift}~\cite{lowe1999object}, followed by finding a homography between both keypoint sets. A
keypoint is a salient image location that is invariant to different
viewing angles and scaling. Keypoints are described by a feature vector. By finding a homography, that is a perspective transformation between the keypoints of the current view and a reference image, the current view can be located in the reference image. The $3 \times 3$
homography matrix ($H$) is based on at least four keypoint matches
between both images. However, usually more points are available,
leading to an overdetermined equation. An initial homography matrix is
then created using a least-squares approach and further refined by
various algorithms.

While this approach is used in frameworks for visual simultaneous localization and mapping, the pipeline of feature detection, description, matching, and pose estimation is computationally complex. Therefore, ground stations for off-board processing or larger processors are usually needed for flight control.  

% TODO:
% Where has the approach been used and is suitable for the proposed algorithm. 