\section{General Discussion}
\label{sec:generaldiscussion}

In this chapter, the results will be discussed with regard to error
statistics, execution frequency, robustness, and scalability. To begin
with, we recapitulate the research questions:

\begin{itemize}
\item R1: Can accurate 2D positions be estimated in real-time, using a
  machine learning-based approach on a limited processor in a
  modifiable indoor environment?
\item R2: Is accurate real-world localization regression or classification
  possible when the training data comprises synthetic data only?
\item R3: Can we predict the goodness of a given map for the proposed
  localization approach?
\end{itemize}

Regarding R1, the conducted experiments provide supportive evidence
that a texton-based machine learning approach is able to accomplish
real-time indoor localization. The proposed algorithm runs with as
frequency of 30\,Hz on a single board computer with limited
CPU. Shifting processing power to an offline training step and relying
on random sampling are the cornerstones for running the algorithm on
processors with limited CPU. Despite the small ratio between extracted
image patches ($s$) and the maximum amount of different image matches
($s*$), the accuracy is hardly affected, and only slightly improves
when incorporating more textons.

Regarding research question R2, the initial idea—to use the synthetically generated images directly as training data—was not successful and not further followed up. This might be also the reason that only few projects have used synthetic images for real-world phenomena. The reality gap between the synthetic data and real-world data was huger than expected. Figure 15 shows an example of two image patches, one synthetically generated, one taken with the camera of the MAV. While the patches can be easily identified as similar for human eyes, the texton maps, where different colors represent different textons, are dissimilar. Blur, lighting settings, and camera intrinsics modify low-level features of the image to a too strong extent. A possible improvement might be to find a mapping from histograms of synthetic images to histograms of real images, by mapping ‘synthetic textons’ to ‘real-world textons’.

Referring to R3, we found some initial evidence that the proposed map evaluation generalizes to the real-world. In contrast to R2, the generalization from the synthetic data to real-world data is of a different nature in this case. The requirement here is that maps that follow the ideal similarity distribution in the synthetically generated images also follow this distribution after being recorded with a camera. Or stated differently, for maps with a low loss value, distant image positions should not have similar histograms using the synthetic images nor the real-world images.  

Despite the overall promising results, we noticed drawbacks of the proposed approach
during the flight tests and directions for future research.

The accuracy—that is the difference between the estimates of the motion tracking system and the texton-based approach—
could be further improved by incorporating more features, for example
histogram of oriented gradients. Investigating further regression
techniques, like Gaussian processes or Bayesian networks that can
inherently handle space and time could be a worthwhile endeavor.

The developed method sets the stage for numerous future research
directions and improvements. The current implementation assumes rather
constant height (up to few centimeters) and no angular rotations of
the MAV. While a quadroter can move in every direction without
performing yaw movements, using the MAV on another vehicle could
require arbitrary yaw movements. In order to limit the complexity of
the dataset, a ``derotation'' of the incoming image could be performed
to align it with the underlying images of the dataset. While the
current approach normalizes each $5\times5$ image patch to unit mean
and zero variance---giving robustness to different lighting
conditions---this procedure could be further extended, for example by
using specific color models.

While the current map evaluation approach used existing fixed images,
it could also serve as a fitness function for an optimization
approach---for example, an evolutionary algorithm---which modifies a
given image. While the solution for a desired loss value of 0 or near 0 might be unique and independent of the original image, a higher loss value might change the initial image only to a certain extent, yielding an “improved version of the image”, which is better suited for the proposed algorithm. 

This could allow to find a near optimal solution for a
given regression technique and give insightful view in the underlying
structure of certain regression techniques.

Currently, \emph{draug} generates image patches based on drawing
samples from parametric distributions. This was motivated by the fact that an ideal map should be independent of previous estimates and based on single images only---ideally requiring no filtering.  
In the future, the
possibility to set flight routes by setting way points above an image could be included. This would allow
to test the ability of the particle filter on synthetic flights.