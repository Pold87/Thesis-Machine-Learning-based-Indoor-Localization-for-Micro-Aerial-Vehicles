\section{Filtering}
\label{sec:filtering}


Computer vision-based estimations are usually noisy or ambiguous, and
so is the proposed system: beginning with the estimations of the
homographies, the ground truth is already based on possibly faulty
labelings. Texton histograms obtained during flight will not perfectly
match the ones in the training data set: blur, lighting settings,
viewing angles and other variables change the shape of the texton
histograms.

To filter out outliers and smooth the estimations, a popular filter
choice is the Kalman filter. However, the Kalman filter is not able to
represent multimodal probability distributions. This makes it rather
unsuitable for the presented approach: if the $k$NN model outputs two
predictions, one would need to use average of these predictions and
feed this value to the Kalman Filter. This approach can lead to biased
predictions, especially, if the the model outputs belong to distant
locations--- to similar texton distributions at these positions.

Instead, the more powerful general Bayesian filter could simultaneously keep
account of both possible locations and resolve the ambiguity as soon
as one location can be favored. In this case, the predictions of the
$k$ neighbors are directly fed into the particle filter without
averaging them first. The filter is able to smooth the estimations,
handle uncertainty, and simultaneously keep track of several competing
position estimations. Since the calculation of a full Bayesian filter is computationally intractable, a particle filter which is based on sampling was used as approximation. 

In general, particle filters estimate the posterior probability of the
state given observations. Specifically, this means that finding the
position of the UAV can be described as $p(X_t \mid Z_t)$, where $X_t$
is the state vector at time $t$ ($x,y$-position, heading, speed,
acceleration) and $Z_t$ are the measurements ($z_1, ..., z_t$, where
each $z_i$ represents the $x,y$ output of the proposed algorithm) up
to time $t$. The state vector is \emph{hidden}, since the variables
cannot be measured directly, therefore the situation can be described
using a hidden Markov model. Instead, noisy or ambiguous data can be
obtained through the proposed algorithm.

The weighted particles are a discrete approximation of the posterior
probability function ($pdf$) of the state vector.

Particle filters have several advantages. First, one can represent
uncertainty by the variance of the state variables of the
particles. Second, the particle filter allows for \emph{sensor fusion}, and
can integrate IMU data or optical flow into the position estimation.
A major disadvantage is the rather high computational complexity. This
can be circumvented by reducing the amount of particles (trading off
speed and accuracy)---allowing for adapting the computational payload
to the used processor.

The used particle filter is initialized using 100 particles at random
$x, y$-positions. To incorporate the distances, the sensor model $p(z
\midx) = \frac{p(x \mid z)p(z)}{p(x)}$ is used, where $\textbf{x} =
((x_1, y_1), (x_2, y_2), \ldots, (x_k, y_k))^T$. A two-dimensional
Gaussian model was used for each point. The parameters of the Gaussian
have been determined by comparing of the positions based on the motion
tracking system with the predictions of the proposed system. This
results in values for the variances in $x$ and $y$, the correlation
$\rho$ between $x$ and $y$. The mean values $\mu$ were set to zero
(no-systematic bias). Figure~\ref{fig:measurementmodel} shows the
results of one such evaluation.