\section{Filtering and Smoothing}
\label{sec:filtering}


Computer vision-based estimations are usually noisy or ambiguous, and
so is the proposed system: beginning with the estimations of the
homographies, the ground truth is already based on possibly faulty
labelings. Texton histograms obtained during flight will not perfectly
match the ones in the training data set: blur, lighting settings,
viewing angles and other variables change the shape of the texton
histograms.

To filter out outliers and smooth the estimations, a popular filter
choice is the Kalman filter. However, the Kalman filter is not able to
represent multimodal probability distributions. This makes it rather
unsuitable for the presented approach: if the $k$NN model outputs two
predictions, one would need to use average of these predictions and
feed this value to the Kalman Filter. This approach can lead to biased
predictions, especially, if the the model outputs belong to distant
locations--- to similar texton distributions at these positions.

Instead, the more powerful general Bayesian filter could simultaneously keep
account of both possible locations and resolve the ambiguity as soon
as one location can be favored. In this case, the predictions of the
$k$ neighbors are directly fed into the particle filter without
averaging them first. The filter is able to smooth the estimations,
handle uncertainty, and simultaneously keep track of several competing
position estimations. Since the calculation of a full Bayesian filter is computationally intractable, a particle filter which is based on sampling was used as approximation. 

In general, particle filters estimate the posterior probability of the
state given observations. Specifically, this means that finding the
position of the UAV can be described as $p(X_t \mid Z_t)$, where $X_t$
is the state vector at time $t$ ($x,y$-position, heading, speed,
acceleration) and $Z_t$ are the measurements ($z_1, ..., z_t$, where
each $z_i$ represents the $x,y$ output of the proposed algorithm) up
to time $t$. The state vector is \emph{hidden}, since the variables
cannot be measured directly, therefore the situation can be described
using a hidden Markov model. Instead, noisy or ambiguous data can be
obtained through the proposed algorithm.

The weighted particles are a discrete approximation of the posterior
probability function ($pdf$) of the state vector.

Particle filters have several advantages. First, one can represent
uncertainty by the variance of the state variables of the
particles. Second, the particle filter allows for \emph{sensor fusion}, and
can integrate IMU data or optical flow into the position estimation.
A major disadvantage is the rather high computational complexity. This
can be circumvented by reducing the amount of particles (trading off
speed and accuracy)---allowing for adapting the computational payload
to the used processor.

The used particle filter is initialized using 100 particles at random
$x, y$-positions. To incorporate the distances, the sensor model $p(z
\midx) = \frac{p(x \mid z)p(z)}{p(x)}$ is used, where $\textbf{x} =
((x_1, y_1), (x_2, y_2), \ldots, (x_k, y_k))^T$. A two-dimensional
Gaussian model was used for each point. The parameters of the Gaussian
have been determined by comparing of the positions based on the motion
tracking system with the predictions of the proposed system. This
results in values for the variances in $x$ and $y$, the correlation
$\rho$ between $x$ and $y$. The mean values $\mu$ were set to zero
(no-systematic bias). Figure~\ref{fig:measurementmodel} shows the
results of one such evaluation.

\begin{figure}[h!]
\begin{center}
\includegraphics[width=0.7\columnwidth]{figures/measurement_model/default_figure}
\caption{{\label{fig:measurementmodel} Measurement model showing the delta x and delta y
    positions%
}}
\end{center}
\end{figure}

This allows to make use of the information in all $k$ neighbors and
keep track of a multimodal distribution. While keeping track of a
multimodal distribution allows for incorporating several possible
states of the system, the problem arises of which mode is the best
one. Using a weighted average of the modes would again introduce the
problem, that the weighted average falls into a low density
region. Therefore, the maximum a posteriori estimate, as described in
\cite{driessen2008map} is used. This approach uses the following
formula to obtain the MAP estimate:

Therefore, the final position estimate is equal to the position of one
of the particles.

\begin{align}
  s_k^{MAP}  &= \argmax_{s_k}{p(s_k \mid Z_k)}\\
             &= \argmax_{s_k}{p(z_k \mid s_k) p(s_k \mid Z_{k-1})} 
\end{align}

Therefore, the MAP estimator in our case is

\begin{align}
s_k^{MAP} = \argmax_{s_k} \lambda^M(s_k)
\end{align}
with

\begin{align}
\lambda^M(s_k) = p(z_k \mid s_k) \sum_{j=1}^Mp(s_k \mid s_{k-1}^j)w^j_{k-1}
\end{align}

This function is now only evaluated at a finite, chosen number of
states, the particles, using

\begin{align}
\hat{s}_k^{MAP} = \argmax_{s_k \in \{s_k^i \mid i=1,\ldots,N\}} \lambda^M(s_k)
\end{align}

In this formula, $p(z_k \mid s_k)$ is the likelihood of the particle
$s_k$ given the current measurement $z_k$. In our setting, this
probability is equal to the weight of the particle $z_k$, therefore
$p(z_k \mid s_k) = w^i_k$.

The estimation of \emph{uncertainty} is a core part of the proposed
approach, being important for safety and accuracy. Therefore,
uncertainty was modeled using the spread of the particles.

In every time step, the particles of the filter get updated based on
the optical flow estimates. These estimates are noisy, as illustrated
in Figure~\ref{fig:edgeflow}. Additionally, optical flow estimates
aggregate noise over time, since each estimate is dependent on the
previous one, leading to drift (\emph{relative position
  estimates}). In contrast, the machine learning-based method makes
independent predictions. While this allow for avoiding accumulating
errors, the predictions do not dependent on each other, and might be
`jumping' between two points. To combine the advantages of both
methods, and leverage out the disadvantages, the particle filter is
used.

An idea was to include the similarity to the neighbors as confidence
value, thus reducing the measurement noise, if a high similarity
between current histogram and a training histogram is
achieved. However, we found no correlation between these
variables. Figure~\ref{fig:cor_sim_measurement} displays the
dependence structure.