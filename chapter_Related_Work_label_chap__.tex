\chapter{Related Work}
\label{chap:relatedwork}

This chapter will discuss advantages and disadvantages of different approaches for indoor localization and
contrast them to the proposed method. While there is a wide range of methods for indoor localization---from laser range scanners over depth cameras to RFID tag based localization---only methods that use the same technical setup (a monocular camera) are discussed. Two types of localization are distinguished: local techniques and global techniques~\cite{fox1999monte}. Local techniques need an initial reference point and estimate a robot's coordinates based on the change in position over time. Once they lost track, the robot's position can typically not recovered. The approaches also suffer from ``drift'' since errors are accumulating over time. Global techniques are more powerful and do not need an initial reference point. They can recover when temporarily losing track and address the \emph{kidnapped robot problem}, in which a robot is carried to an arbitrary location~\cite{engelson1992error}.  

In general, comparing the accuracy and run-time of different localization methods is difficult: target systems and test environments are often too different to draw comparisons. The annual Microsoft Indoor Localization Competition\footnote{\url{https://www.microsoft.com/en-us/research/event/microsoft-indoor-localization-competition-ipsn-2016/}} aims at setting a standardized testbed for comparing near real-time indoor location technologies. However, since the competition does not require lightweight platforms and allows for using external infrastructure such as WiFi routers, no vision-only approach was presented at the competition yet.