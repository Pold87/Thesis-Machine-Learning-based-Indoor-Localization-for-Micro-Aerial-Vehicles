\chapter{Related Work}
\label{chap:relatedwork}

This chapter will discuss advantages and disadvantages of different approaches for indoor localization and
contrast them to the proposed method. While there is a wide range of methods for indoor localization---from laser range scanners over depth cameras to RFID tag based localization---only methods that use the same technical setup (a monocular camera) are discussed. Two types of localization are distinguished: local techniques and global techniques. Local techniques estimate a robot's coordinates based on the change in position over time. The approaches need an initial reference point. Once they lost track, the robot's position can typically not recovered. 


In general, comparing the accuracy and run-time of different localization methods is difficult: target systems and test environments are often too different to draw comparisons. The annual Microsoft Indoor Localization Competition\footnote{\url{https://www.microsoft.com/en-us/research/event/microsoft-indoor-localization-competition-ipsn-2016/}} aims at setting a standardized testbed for comparing near real-time indoor location technologies. However, since the competition does not require lightweight platforms and allows for using external infrastructure such as WiFi routers, no vision-only approach was presented at the competition yet.