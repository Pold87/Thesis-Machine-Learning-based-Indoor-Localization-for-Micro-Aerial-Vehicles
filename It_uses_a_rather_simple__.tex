 It uses a rather simple stimulus-response behavior to
estimate the position, which circumvents the requirement to store a
map in the UAV's `mind'. To assign $x,y$-coordinates to images in a
training set, keypoints in the current image and a map image are
detected. This is then followed by finding a homography between
them. In the next step, the complexity of these images is reduced by
\emph{sparse encoding}: images are represented by determining their
histogram of textons---small characteristic image
patches~\cite{varma2005statistical}.
New images can then also be encoded by texton histograms and matched
to images with known $x,y$-positions using the $k$-Nearest Neighbors
($k$NN) algorithm. The computational effort of the approach can be
adjusted by modifying the amount of extracted patches, resulting in a
trade-off between accuracy and execution frequency. While the $k$NN
algorithm is one of the simplest machine learning algorithms, it
offers several advantages: it is non-parametric, allowing for the
modeling of arbitrary distributions. Its capability to output multiple
predictions with an associated confidence allows for neat integration
with the proposed particle filter. Finally, computational complexity
can be modified by changing the size of the datasets.