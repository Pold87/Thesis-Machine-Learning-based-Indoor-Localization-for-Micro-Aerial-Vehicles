\section{Pillar III: Map evaluation}
\label{sec:mapeval}

\subsection{Synthetic Data Generation}
\label{sec:syntheticdatageneration}

In the scope of this thesis, an application to simulate different
camera positions during flight was created. It generates synthetic
image patches based on perspective transformations of a map
image. Examples of generated images are displayed in
Figure~\ref{fig:montage}. The application allows for comparing and
predicting the performance of different maps. The software is written
in C++ and OpenCV~3.0.0.

The software is able to generate a specified amount of image patches
using random values for rotational angles, translational shifts, as
well as blur, contrast and brightness intensity. The values are
sampled from different probability distributions, see
Table~\ref{tab:distributions} for a summary. An additional graphical
user interface (GUI) displays the result of applied transformations
and saves the generated images.

To simulate camera movements in 3D space, a 2D to 3D projection of the
image is performed first. Then, by building separate rotation matrices
$R_x$, $R_y$, and $R_z$ around the axes $x$, $y$, and $z$, the
rotations can be performed separately. Next the rotation matrix $R$ is
created by multiplying the separate matrices, i.e.,
$R = R_x \times R_y \times R_z$. The 3D translation matrix is
multiplied by the transposed rotation matrix. This step is crucial to
rotate the \emph{camera model} and not the image itself. Finally,
after performing all steps, a projection from 3D space to 2D is
applied, to obtain the transformed image.

\begin{table}[h!]
  \centering
  \begin{tabular}{llllll}
    \toprule
    \multicolumn{6}{c}{Distribution}                                                         \\
    \multicolumn{3}{c}{Uniform ($\mathcal{U}$)} & \multicolumn{3}{c}{Normal ($\mathcal{N}$)} \\
    \cmidrule(r){1-3}\cmidrule(r){4-6}
    Parameter                                   & Min   & Max   & Parameter  & M    & STD    \\
    \cmidrule(r){1-3}\cmidrule(r){4-6}
    Yaw                                         & $0$   & $360$ & Roll       & $90$ & $3$    \\
    Translation X                               & $100$ & $500$ & Pitch      & $90$ & $4$    \\
    Translation Y                               & $100$ & $500$ & Brightness & $2$  & $0.1$  \\
    Height                                      & $100$ & $700$ &            &      &        \\
    Blur                                        & $1$   & $10$  &            &      &        \\
    Contrast                                    & $2$   & $3$   &            &      &        \\
    \bottomrule
  \end{tabular}
  \caption[Distributions for the different
  parameters of the synthetic data augmentation tool.]{The table shows the used distributions for the different
    parameters of the synthetic data augmentation tool.}
  \label{tab:distributions}

\end{table}

\begin{figure}[h!]
\begin{center}
\includegraphics[width=0.7\columnwidth]{figures/montage/default_figure}
\caption{{\label{fig:montage} 16 example images generated by the synthetic data generation
    tool. The black parts are the parts beyond the image borders of
    the underlying image (i.e., the part where pixel maps to) or the
    ones that could not be identified during the mosaic making
    process.%
}}
\end{center}
\end{figure}

\subsection{Evaluation Scheme}
\label{sec:evaluationscheme}

The performance of a method might largely depend
on the environment it is used in. The evaluation of a map is
difficult, since the obtained histograms during a real flight depend
on many factors: motion blur, distance to the map and rotations
proportional to the map.

Therefore, we propose an initial evaluation scheme for given
maps. This scheme assigns a global fitness value to a given map,
proportional to the expected accuracy if it is used in the physical
world. Additionally, it allows to inspect the given map and detect the
regions that are responsible for the overall fitness value.

In the first step of the map evaluation procedure, $N$ different
patches of a given map are generated using the tool \emph{draug}
(Section~\ref{sec:draug}). We propose the following loss function
($L$) for evaluating a given map ($\mathcal{M}$):

\begin{align}
  L(\mathcal{M}) &= \sum_{i = 1}^{N} \sum_{j = 1}^{N} \ell(d_a(h_i, h_j), d_e(h_i, h_j))
\end{align}

\begin{align}
  \ell(x, y) &= x - y\\
  d_a(h_i, h_j) &= \text{cosine\_similarity}(h_i, h_j)\\
  d_e(h_i, h_j) &= f_X(pos_i) = f_X(x_i, y_i)\\
\end{align}

\begin{align}
\mu = pos_j = (x_j, y_j)\\
\Sigma =
  \begin{bmatrix}
    \rho & 0\\
    0 & \rho\\
  \end{bmatrix}
\end{align}

The idea behind the global loss function $L$ is that histograms in closeby areas
should be similar and the similarity should decrease the further away
two positions are. This is modeled as a 2-dimensional Gaussian with 0
covariance (Figure~\ref{fig:model}). The variance is depended on the
desired accuracy ($\rho$): the lower the variance, the more punctuated
a certain location is but also the higher the risk that a totally
wrong measurement occurs. The following visualization are based on
color histograms (and not texton histograms) for easier visual
analysis.

\begin{figure}[h!]
\begin{center}
\includegraphics[width=0.7\columnwidth]{figures/model-crop/default_figure}
\caption{{\label{fig:model} Ideal
    histogram similarity for a given position. Histograms taken at
    positions close to $\textbf{x} = (400, 300)$ should be similar to
    this histogram. The further away the position of a certain
    histogram, the lower the ideal similarity should be.%
}}
\end{center}
\end{figure}

\begin{figure}[h!]
\begin{center}
\includegraphics[width=0.7\columnwidth]{figures/mosaic_enlarged/default_figure}
\caption{{\label{fig:map} This figure shows a map with a repeating pattern: two
    yellow rectangles.%
}}
\end{center}
\end{figure}