\chapter{Conclusion and Future Work}
\label{chap:conclusion}

This thesis presented a novel approach for fast and accurate indoor
localization of MAVs. We pursued an onboard design without the need of
an additional ground station to foster flexibility and autonomy. The
conducted on-ground and in-flight experiments underline the real-world
applicability of the system. Promising results were obtained for
waypoint navigation, accurate landing, and stable hovering in the
indoor environment. The
used approach is based on three pillars.\\
The first pillar shifts computational effort from the flight phase to
an offline preprocessing step. This allows for using sophisticated
algorithms, without affecting performance during flight.\\
The second pillar states that during flight the MAV lightweight
algorithms should run with low-performing processors. These algorithms
should be able to trade off speed with accuracy. This allows to use
them on a wide range of models, from pico-drones to MAVs with a
wing-span of over one meter. Examples of these adaptable algorithms
are the particle filter
and the texton-based approach. \\
The third pillar is the known---and possibly---modifiable
environment. This knowledge and flexibilty allows to predict the
quality of the used approach. In contrast to SLAM frameworks, in which
the task is to simultaneous mapping and localization, the proposed
approach is intended for various repetitive indoor activities.
