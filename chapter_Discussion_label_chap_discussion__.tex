\chapter{Discussion}
\label{chap:discussion}

The comparison of the amount of samples with the average cosine
similarity has shown that only a small part of the maximum amount of
samples suffices to achieve similarities larger than $99\,\%$. In
fact, $\frac{400}{640 \times 480} = 0.13\,\%$. This set the stage for
large speed-ups during live operation. Additionally, it allow for further speed-ups depending on the processing power of the used
CPU.

The evaluation of different maps using the synthetic data showed
considerable differences between the evaluated images. The range of
losses from 0.24 to 0.99 clearly shows the different suitabilities of different maps for the proposed algorithm. In order to visually evaluate the
proposed map evaluation technique, a simple map was constructed with
two repeating tiles. The image with the minimum and the one with the
maximum loss value based on their color histogram are shown in
Figure~\ref{fig:minmaximg}. The different patterns of the images are
clearly visible: while the image with the minimum value fulfills the
desired properties---closeby areas have similar color values, distant
areas are dissimilar, the image with the maximum loss is mainly black
resulting in similar histograms all over the place and leading to
high loss values. This initial evidence can be taken to test the
predictive power of the evaluation algorithm for texton histograms.
