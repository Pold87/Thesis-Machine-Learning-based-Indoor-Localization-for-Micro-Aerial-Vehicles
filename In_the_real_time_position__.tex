In the real-time position estimation experiment (Experiment~\ref{sec:experiment-real-time}), the initial mean error was rather large: 130\,cm in $x$-direction and 90\,cm in $y$-direction. A more in-depth analysis revealed that the estimates of the particle filter were lacking behind the position estimates of the motion tracking system. This was due to the simple motion model of the particle filter that was based on Gaussian noise only. By shifting the estimates of the particle filter by six frames, that is approx. 0.46 seconds at a frequency of 13\,Hz, the error could be reduced to 46\,cm in $x$-direction and 54\,cm in $y$-direction. This lag can be addressed by to strategies: (i) slower speed during flight, (ii) a better or more flexible motion model. 