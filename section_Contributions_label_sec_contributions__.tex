\section{Contributions}
\label{sec:contributions}

The first contribution of this thesis is a machine learning-based
indoor localization system that is able to run in real-time on board
of an MAV, paving the way to a fully autonomous system. In contrast to
existing \emph{active} approaches, the proposed \emph{passive}
approach only uses a single, monocular, downward-looking camera. Since
computer vision-based localization approaches yield noisy estimates, a
variant of a particle filter was developed that aggregates estimations
over time to produce more accurate predictions. It handles the
estimates of the $k$NN algorithm in an integrative way and resolves
position ambiguities. The method is an absolute position system and does not suffer from error
accumulation over time.

The second contribution is a map evaluation technique that predicts
the suitability of a given environment for the proposed algorithm. To
this end, a synthetic data generation tool was developed that creates
random variations of an image. The tool simulates different viewing
angles, motion blur, and lighting settings; the generated synthetic
images are labeled with $x,y$-coordinates based on the 3D position of
the simulated camera model.

The developed software is made publicly available. It encompasses (i)
the localization algorithm as part of the Paparazzi project, which
consists of the texton-based approach in combination with a particle
filter (ii) software for augmenting an image with synthetic views,
(iii) a script for evaluating a map based on histograms and
corresponding $x,y$-positions.

