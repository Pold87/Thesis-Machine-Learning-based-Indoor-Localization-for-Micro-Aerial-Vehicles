\subsection{Fiducial Markers}
\label{sec:fiducialmarkers}

Fiducial markers, which are often used in augmented reality
applications~\cite{kato1999marker,garrido2014automatic}, have been
used for UAV localization and
landing~\cite{eberli2011vision,bebop2015}.
The markers encode information by the spatial arrangement of
black and white or colored image patches, and their corners can be
used for estimating the camera pose at a high frequency, using
triangulation methods. The positions of the markers in
an image are usually determined using local thresholding, which identifies salient image
regions. The positions are further refined by removing improbable shapes,
yielding an adjusted version of possible marker positions. Figure~\ref{fig:aruco} shows examples of fiducial markers from the ArUco library~\cite{aruco2014}.

A drawback of the approach is that, during flight, motion blur and partial occlusion can hinder the
detection of markers~\cite{albasiouny2015mean}. Moreover, these markers might be considered as
visually unpleasant and may not fit into a product or environmental
design~\cite{chu2013halftone}. They offer little
flexibility, since one has to rely on predefined marker dictionaries.
There is no direct way to modify the time complexity of the algorithm.

% TODO: which errors have been achieved