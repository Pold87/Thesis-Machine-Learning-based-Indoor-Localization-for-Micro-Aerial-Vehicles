\chapter{Introduction}
\label{chap:introduction}

Many crimes over the last decades have been solved thanks to footage
captured by surveillance cameras. However, stationary cameras can be
easily manipulated or avoided once it is known where they are
located. One possible solution may be the use of micro aerial vehicles
(MAVs) for surveillance. To date, indoor employment of these vehciles
is still hindered by several limitations. The focus of this thesis is,
thus, the development of accurate and fast indoor localization for
MAVs combining computer vision and machine learning techniques.

Since precision, reliability, and rigorous error avoidance are crucial
for safe flight, autonomous indoor navigation of an MAV is a
challenging task. While unmanned aerial vehicles (UAVs) for
outdoor usage can rely on the global positioning system (GPS), this system is
usually not available in confined spaces and would not provide
sufficiently accurate estimates in cluttered environments.

If sufficient computational and physical power is available, a typical
approach to estimate a UAV's position is by using active laser
rangefinders~\cite{grzonka2009towards,bachrach2009autonomous}.
Although this approach is used in some simultaneous localization and
mapping (SLAM) frameworks, it is usually not feasible for MAVs because
they can carry only small payloads. A viable alternative are passive
computer vision techniques. Relying on visual information scales down
the physical payload since cameras are often significantly lighter
than laser
rangefinders~\cite{blosch2010vision,angeli20062d,ahrens2009vision}.
Additionally, many commercially available drones are already equipped
with cameras. In contrast to other existing approaches, the proposed algorithm does not rely on data from the inertial measurement unit (IMU). The only required tool is a camera, which makes the proposed algorithm rather safe to failure. Cameras are lightweight and not affected by external, such as magnetic fields. Additionally, relying on reduces the number of possible points of failure.  