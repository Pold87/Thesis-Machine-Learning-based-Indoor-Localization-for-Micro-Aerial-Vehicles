\chapter{Introduction}
\label{chap:introduction}


In the world of automation, micro aerial vehicles (MAVs) provide unprecedented perspectives for domestic and industrial applications. They can serve as mobile surveillance cameras, fast transport platform, or as technical waiter in a restaurant. However, indoor employment of these vehicles
is still hindered by the lack of real-time positions estimates. The focus of this thesis is,
thus, the development of accurate and fast indoor localization for
MAVs combining computer vision and machine learning techniques.

%Many crimes over the last decades have been solved thanks to footage
%captured by surveillance cameras. However, stationary cameras can be
%easily manipulated or avoided once it is known where they are
%located. One possible solution may be the use of micro aerial vehicles
%(MAVs) for surveillance. To date, indoor employment of these vehicles
%is still hindered by several limitations. The focus of this thesis is,
%thus, the development of accurate and fast indoor localization for
%MAVs combining computer vision and machine learning techniques.

Since precision, reliability, and rigorous error avoidance are crucial
for safe flight, autonomous indoor navigation of an MAV is a
challenging task. While unmanned aerial vehicles (UAVs) for
outdoor usage can rely on the global positioning system (GPS), this system is
usually not available in confined spaces and would not provide
sufficiently accurate estimates in cluttered environments.