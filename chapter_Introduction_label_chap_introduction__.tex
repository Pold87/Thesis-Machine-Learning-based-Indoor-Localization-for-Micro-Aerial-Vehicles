\chapter{Introduction}
\label{chap:introduction}

Many crimes over the last decades have been solved thanks to footage
captured by surveillance cameras. However, stationary cameras can be
easily manipulated or avoided once it is known where they are
located. One possible solution may be the use of micro aerial vehicles
(MAVs) for surveillance. To date, indoor employment of these vehciles
is still hindered by several limitations. The focus of this thesis is,
thus, the development of accurate and fast indoor localization for
MAVs combining computer vision and machine learning techniques.

Since precision, reliability, and rigorous error avoidance are crucial
for safe flight, autonomous indoor navigation of an MAV is a
challenging task. While unmanned aerial vehicles (UAVs) for
outdoor usage can rely on the global positioning system (GPS), this system is
usually not available in confined spaces and would not provide
sufficiently accurate estimates in cluttered environments.


Processors of MAVs have only limited processing power. Moreover,
it is intended to further reduce the size of MAVs to that of an
insect, hence necessitating lightweight and scalable position
estimation algorithms.

If sufficient computational and physical power is available, a typical
approach to estimate a UAV's position is by using active laser
rangefinders~\cite{grzonka2009towards,bachrach2009autonomous}.
Although this approach is used in some simultaneous localization and
mapping (SLAM) frameworks, it is usually not feasible for MAVs because
they can carry only small payloads. A viable alternative are passive
computer vision techniques. Relying on visual information scales down
the physical payload since cameras are often significantly lighter
than laser
rangefinders~\cite{blosch2010vision,angeli20062d,ahrens2009vision}.
Additionally, many commercially available drones are already equipped
with cameras. In contrast to other existing approaches, it does not rely on data from the inertial measurement unit (IMU). The only required tool is a camera, which makes the proposed algorithm rather safe to failure. Cameras are lightweight and not affected by external, such as magnetic fields. Additionally, relying on reduces the number of possible points of failure.  

However, this reduced physical payload is not without cost: it must be
traded off against the higher computational payload for the onboard
CPU. For example, a standard technique for 3D pose estimation extracts
keypoints of the current camera image and a map image and then
determines a homography between both keypoint sets. While this
approach has been used for visual SLAM for
MAVs~\cite{blosch2010vision}, the pipeline for accurate feature
detection, description, matching, and pose estimation is
CPU-intensive~\cite{kendall2015convolutional}.
One way to overcome this problem is to process the data on a powerful
external processor by establishing a wireless connection between the
MAV and a ground station. Such off-board localization techniques often
lack the versatility to function in changing environments, though, due
to factors---such as the bandwidth, delay, or noise of the wireless
connection---interfering with the system's reliability.

In the approach proposed in this thesis, computational power will be
shifted to an offline training phase to achieve high-speed during live
operation. In contrast to visual SLAM frameworks, this project considers
scenarios in which the environment is known beforehand or can be even
actively modified. The environment is non-dynamic and planar, therefore, the UAV will make use of texture on the bottom or ceiling of the environment.
 It uses a rather simple stimulus-response behavior to
estimate the position, which circumvents the requirement to store a
map in the UAV's `mind'. To assign $x,y$-coordinates to images in a
training set, keypoints in the current image and a map image are
detected. This is then followed by finding a homography between
them. In the next step, the complexity of these images is reduced by
\emph{sparse encoding} : images are represented by determining their
histogram of textons---small characteristic image
patches~\cite{varma2005statistical}.
New images can then also be encoded by texton histograms and matched
to images with known $x,y$-positions using the $k$-Nearest Neighbors
($k$NN) algorithm. The computational effort of the approach can be
adjusted by modifying the amount of extracted patches, resulting in a
trade-off between accuracy and execution frequency. While the $k$NN
algorithm is one of the simplest machine learning algorithms, it
offers several advantages: it is non-parametric, allowing for the
modeling of arbitrary distributions. Its capability to output multiple
predictions with an associated confidence allows for neat integration
with the proposed particle filter. Finally, computational complexity
can be modified by changing the size of the datasets.


This opens the door for improving the proposed
algorithm by changing the map. On the basis of desired characteristics
of a given map, an evaluation technique was developed that determines
the suitability of an environment for the proposed approach. This
technique allows for spotting distant regions with similar image
features, which could lead to deteriorated performance. The evaluation
can be performed using a given map image or recorded images during
flight. In the former case, synthetic images will be generated from
the map image that simulate images taken during flight.

The goal is to estimate $x,y$-coordinates in real-time and on-board of the UAV. It is assumed that the UAV flies at a approximately constant height, such that the estimation of height is not necessary.


\section{Contributions}
\label{sec:contributions}

The first contribution of this thesis is a machine learning-based
indoor localization system that is able to run in real-time on board
of an MAV, paving the way to a fully autonomous system. In contrast to
existing \emph{active} approaches, the proposed \emph{passive}
approach only uses a single, monocular, downward-looking camera. Since
computer vision-based localization approaches yield noisy estimates, a
variant of a particle filter was developed that aggregates estimations
over time to produce more accurate predictions. It handles the
estimates of the $k$NN algorithm in an integrative way and resolves
position ambiguities. The method operates on single images and, in
contrast to, for example visual odometry, does not suffer from error
accumulation over time.

The second contribution is a map evaluation technique that predicts
the suitability of a given environment for the proposed algorithm. To
this end, a synthetic data generation tool was developed that creates
random variations of an image. The tool simulates different viewing
angles, motion blur, and lighting settings; the generated synthetic
images are labeled with $x,y$-coordinates based on the 3D position of
the simulated camera model.

The developed software is made publicly available. It encompasses (i)
the localization algorithm as part of the Paparazzi project, which
consists of the texton-based approach in combination with a particle
filter (ii) software for augmenting an image with synthetic views,
(iii) a script for evaluating a map based on histograms and
corresponding $x,y$-positions.
