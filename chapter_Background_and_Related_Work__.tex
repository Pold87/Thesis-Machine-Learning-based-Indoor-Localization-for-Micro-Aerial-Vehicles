\chapter{Background and Related Work}
\label{chap:relatedwork}

This chapter will discuss advantages and disadvantages of different approaches for indoor localization and
contrast them to the proposed method. Important
background knowledge for later chapters, concerning feature detection,
image matching, and textons will be introduced.

\section{Vision-based Localization Methods}

\subsection{Fiducial Markers}
\label{sec:fiducialmarkers}

Fiducial markers, which are often used in augmented reality
applications~\cite{kato1999marker,garrido2014automatic}, have been
used for UAV localization and
landing~\cite{eberli2011vision,bebop2015}.

The markers encode information by the spatial arrangement of
black and white or colored image patches, and their corners can be
used for estimating the camera pose at a high frequency, using
triangulation methods.

Existing approaches usually identify markers in
an image using local thresholding, which identifies salient image
regions. These are further refined by removing improbable shapes,
yielding an adjusted version of possible marker positions.

During flight, motion blur and varying distances can hinder the
detection of markers. Moreover, these markers might be considered as
visually unpleasant and might not fit into a product or environmental
design~\cite{chu2013halftone}. Additionally, they offer little
flexibility, since one has to rely on predefined marker dictionaries.
Their is no direct way to modify the time complexity of the algorithm:
Reducing the resolution of the images is a time-intense method.

\subsection{Homography Determination \& Keypoint Matching}
\label{sec:keypointmatching}

A standard approach for estimating camera pose is detecting and
describing keypoints of the current view and a reference image, using
algorithms such as \textsc{Sift}~\cite{lowe1999object}. A
keypoint is a salient image location that is invariant to different
viewing angles and scaling. By finding a homography, that is a perspective transformation between the image keypoints, it can be determined where
the current view is located in the reference image. The $3 \times 3$
homography matrix ($H$) is based on at least four keypoint matches
between both images. However, usually more points are available,
leading to an overdetermined equation. An initial homography matrix is
then created using a least-squares approach and further refined by
various algorithms.

% TODO:
% Where has the approach been used and is suitable for the proposed algorithm. 


\subsection{Optical Flow}
\label{sec:opticalflow}

Optical flow estimates the ‘flow’ between successive images: the shift of corresponding image keypoints in $x,y$-direction.
The approach is computationally rather complex. It is a relative localization method: without an initial reference point, it can only estimate the distance to an initial reference point but not its absolute position in space. The approach suffers from drift: since the distance travelled is based on comparing successive images, errors are accumulating over time. The approach alone does not have the  the possibility to correct these errors.


\section{Texton-based Approaches}
\label{sec:textonbasedapproaches}

Textons are small characteristic image patches that can be used as image features.
\citeauthor{varma2005statistical} originally introduced textons for
classifying different textures, showing that they outperform
computationally more complex algorithms, like Gabor
filters~\cite{varma2005statistical}.

Instead of convolving a window over the entire image, the filters can be applied at randomly sampled image positions, leading to similar texton histograms compared to the histograms when full sampling is used. The choice of the number of samples allows for modifying the computational effort, resulting in a
trade-off between accuracy and execution frequency. A disadvantage is
that it discards all information about the spatial arrangement of
textons---it does not make use of the \emph{Where} of the information,
just of the \emph{What}, which might result in different areas with
similar histograms.

\citeauthor{de2009design} use the texton approach for distinguishing
between three height classes during flight~\cite{de2009design}. Using
a nearest neighbor classifier, their approach achieves a height
classification error of approximately 22\,\% on a hold-out test set.
This enables a flapping-wing MAV during an experiment to roughly hold
its height. In another work, \citeauthor{de2012appearance} introduce
the \emph{appearance variation cue} for estimating the proximity to
objects~\cite{de2012appearance}. Since closer objects should have less
variation, these objects should appear less varied. Using this method,
their MAV is successfully able to avoid obstacles in a $5m \times 5m$
office space and achieves an AUC of up to .97 on rather distorted
images. Additionally, it performs better than or, at least, equal to
optical flow estimations.
 

\section{Synthetic Data Generation}
\label{sec:syntheticdatageneration}

Acquiring data for machine learning methods is usually time-consuming. A viable alternative seems to be the 
use of synthetic data. However, their applicability in real-world situations is still hindered by
the reality gap---the difference between simulations and
real-world situations. The gap can manifest itself in different
lighting conditions, resolutions, or blur or process noise, like wind,
or air resistance.
