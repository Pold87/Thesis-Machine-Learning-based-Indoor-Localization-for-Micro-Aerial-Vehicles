\chapter{Background and Related Work}
\label{chap:relatedwork}

This chapter will discuss advantages and disadvantages of different approaches for indoor localization and
contrast them to the proposed method. Important
background knowledge for later chapters, concerning feature detection,
image matching, and textons will be introduced. While the range of methods for indoor localization is wide-spread---from laser range scanners over depth cameras to RFID tag based localization---only methods that are closely related to the proposed approach are discussed. In detail, this means that the approach should be based on a monocular camera.

\section{Vision-based Localization Methods}

\subsection{Fiducial Markers}
\label{sec:fiducialmarkers}

Fiducial markers, which are often used in augmented reality
applications~\cite{kato1999marker,garrido2014automatic}, have been
used for UAV localization and
landing~\cite{eberli2011vision,bebop2015}.
The markers encode information by the spatial arrangement of
black and white or colored image patches, and their corners can be
used for estimating the camera pose at a high frequency, using
triangulation methods. The positions of the markers in
an image are usually determined using local thresholding, which identifies salient image
regions. The positions are further refined by removing improbable shapes,
yielding an adjusted version of possible marker positions.

During flight, motion blur and varying distances can hinder the
detection of markers. Moreover, these markers might be considered as
visually unpleasant and might not fit into a product or environmental
design~\cite{chu2013halftone}. They offer little
flexibility, since one has to rely on predefined marker dictionaries.
There is no direct way to modify the time complexity of the algorithm.

% TODO: which errors have been achieved

\subsection{Homography Determination \& Keypoint Matching}
\label{sec:keypointmatching}

A standard approach for estimating camera pose is detecting and
describing keypoints of the current view and a reference image, using
algorithms such as \textsc{Sift}~\cite{lowe1999object}, followed by finding a homography between both keypoint sets. A
keypoint is a salient image location that is invariant to different
viewing angles and scaling. Keypoints are described by a feature vector. By finding a homography, that is a perspective transformation between the keypoints of the current view and a reference image, the current view can be located in the reference image. The $3 \times 3$
homography matrix ($H$) is based on at least four keypoint matches
between both images. However, usually more points are available,
leading to an overdetermined equation. An initial homography matrix is
then created using a least-squares approach and further refined by
various algorithms.

% TODO:
% Where has the approach been used and is suitable for the proposed algorithm. 

\subsection{Optical Flow}
\label{sec:opticalflow}

Optical flow estimates the motion based on successive images: the shift of corresponding image keypoints in $x,y$-direction.
The approach is computationally rather complex. It is a relative localization method: without an initial reference point, it can only estimate the distance to an initial reference point but not its absolute position in space. The approach suffers from drift: since the distance travelled is based on comparing successive images, errors are accumulating over time. The approach alone does not have the  the possibility to correct these errors.
