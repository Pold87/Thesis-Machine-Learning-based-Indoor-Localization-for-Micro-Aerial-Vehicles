However, this reduced physical payload is not without cost: it must be
traded off against the higher computational payload for the onboard
CPU. Vision-based position estimation is usually a time-consuming and memory-intense procedure. 
For example, a standard technique for 3D pose estimation extracts
keypoints of the current camera image and a map image and then
determines a homography between both keypoint sets. While this
approach has been used for visual SLAM for
MAVs~\cite{blosch2010vision}, the pipeline for accurate feature
detection, description, matching, and pose estimation is
CPU-intensive~\cite{kendall2015convolutional}.
One way to overcome this problem is to process the data on a powerful
external processor by establishing a wireless connection between the
MAV and a ground station. Such off-board localization techniques often
lack the versatility to function in changing environments, though, due
to factors---such as the bandwidth, delay, or noise of the wireless
connection---interfering with the system's reliability.