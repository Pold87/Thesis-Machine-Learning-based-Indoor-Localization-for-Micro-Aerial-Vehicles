For creating the orthomap, the images from the dataset have to be stichted
together to get a hyperspatial image of the scene. By orthorectifying
these pictures, different camera angles can be straightened out,
yielding seamless stitches. This results in a map of the
environment. The stitched image has a higher resolution than the
single images and contains a greater range of detail. The stitching
step includes challenges: a subset of the recorded images might be
distorted and each image must be orthorectified, otherwise perspective
transformations could hinder the stitching process. The images can be
orthorectified by estimating the most probable viewing angle based on
the set of all images. Since a downward-looking camera is attached to
the UAV, most images will be roughly aligned with the z-axis, given
slow flight.